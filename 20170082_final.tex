\documentclass{article}
\usepackage{amsmath}
\usepackage{listings}
\usepackage{xcolor}
\usepackage{graphicx}
\usepackage[a4paper, total={6in, 8in}]{geometry}


\definecolor{codegreen}{rgb}{0,0.6,0}
\definecolor{codegray}{rgb}{0.5,0.5,0.5}
\definecolor{codepurple}{rgb}{0.58,0,0.82}
\definecolor{backcolour}{rgb}{0.95,0.95,0.92}

\lstdefinestyle{mystyle}{
    backgroundcolor=\color{backcolour},   
    commentstyle=\color{codegreen},
    keywordstyle=\color{magenta},
    numberstyle=\tiny\color{codegray},
    stringstyle=\color{codepurple},
    basicstyle=\ttfamily\footnotesize,
    breakatwhitespace=false,         
    breaklines=true,                 
    captionpos=b,                    
    keepspaces=true,                 
    numbers=left,                    
    numbersep=5pt,                  
    showspaces=false,                
    showstringspaces=false,
    showtabs=false,                  
    tabsize=2
}

\lstset{style=mystyle}
\setlength\parindent{0pt}


\begin{document}

\author{Dongwon Kim, Akotet Yeshaw Tesema, Eihyuk Moon, Belay Zeleke, }
\title{HSS310 Final}

\maketitle

% Please note that this includes Dongwon Kim's individual work.

\section{Part 1}
\subsection{Problem 1}

\hspace{1em} Suppose there are \(n\) regions each of which consist of a firm and a household in the economy.
The firms' production function and the households' utility and the law of motion for capital follows the same form as in problem 7.
The total factor productivity of the firm in region \(i\) is \(z_i\) follows the AR(1) process with \(\rho_i\).
The total factor productivity is pairwise independent.
Suppose there is a goverment in the economy that taxes each firm's production and redistributes it into each region in an egalitarian manner.
We can expect that the government's role will allow the value function to be the same regardless of the distribution of capital stock and the total factor productivity.
In this problem, we prove that this is in fact the case.\\ \\
The Bellman equation for this problem is written as:
\begin{align*}
    V(\mathbf{k}_t, \mathbf{z}_t) &= \max_{\mathbf{k}_{t+1}, \mathbf{n}_{t}, \mathbf{t}_{t}, \mathbf{g}_{t}} \{ \sum_{i} u_{i}\left(z_{i,t} k_{i,t}^\alpha n_{i,t}^{1-\alpha} -t_{i, t} + g_{i, t} - k_{i, t+1}, 1 - n_{i,t}\right) \\
    &+ \beta E[V(\mathbf{k}_{t+1}, \mathbf{z}_{t+1}) | I_t] \}
\end{align*}
Where \(t_i\) is the tax rate for the firm in region \(i\), and \(g_i\) is the transfer payment for the households in region \(i\).
\begin{equation}
    \text{subject to } \sum_{i} t_{i, t} = \sum_{i} g_{i, t}
\end{equation}
This equation implies that \( \sum_{i} s_{i,t} = 0 \) where \( -\mathbf{t}_{t} + \mathbf{g}_{t} = \mathbf{s}_{t}\), which is a preservation of the total tax revenue and the total transfer payments.\\ \\
This substitution simplfies the bellman equation into:
\begin{align*}
    V(\mathbf{k}_t, \mathbf{z}_t) &= \max_{\mathbf{k}_{t+1}, \mathbf{n}_{t}, \mathbf{t}_{t}, \mathbf{g}_{t}} \{ \sum_{i} u_{i}\left(z_{i,t} k_{i,t}^\alpha n_{i,t}^{1-\alpha} + s_{i, t} - k_{i, t+1}, 1 - n_{i,t}\right) \\
    &+ \beta E[V(\mathbf{k}_{t+1}, \mathbf{z}_{t+1}) | I_t] \}
\end{align*}
The Euler equation for region \(i\) is:
\begin{equation}
    \frac{1}{c_{i, t}} =  \frac{\beta}{c_{i, t+1}} \alpha k_{i, t+1}^{\alpha - 1} n_{i, t+1}^{1-\alpha} E\left[z_{i, t+1} | I_t\right]
\end{equation}

The first order condition with respect to \(n_{i, t}\) with envelope condition is:
\begin{equation}
    \frac{\chi}{1-n_{i,t}} = \frac{1}{c_{i,t}} \cdot z_{i,t} (k_{i,t})^{\alpha}(1-\alpha)n_{i,t}^{-\alpha}
\end{equation}

The gradient with respect to \(\mathbf{s}_t\), we get:
\begin{equation}
    0 = \sum_{i} u_{i, c}\left(z_{i,t} k_{i,t}^\alpha n_{i, t}^{1-\alpha} +s_{i, t} - k_{i, t+1}, 1 - n_{i, t}\right) ds_{i, t}
\end{equation}
In other words:
\begin{equation}
    0 = \sum_{i} \frac{1}{c_{i,t}} ds_{i, t}
\end{equation}
Since \( \sum_{i} s_{i,t} = 0 \), we have \( \sum_{i} ds_{i, t} = 0 \). These equations along with \( 0 = \sum_{i} \frac{1}{c_{i,t}} ds_{i, t} \) implies
\begin{equation}
    \frac{1}{c_{i,t}} = \frac{1}{c_{j,t}} \quad \forall i, j
\end{equation}

Apply the guess-and-verify method, where \(V(\mathbf{k}_t, \mathbf{z}_t) = A + \sum_{i} B_i \log k_{i, t} + \sum_{i} C_i \log z_{i, t}\), then the Bellman equation is written as:
\begin{align*}
    A +  \sum_{i} B_i \log k_{i, t} + \sum_{i} C_i \log z_{i, t} = & \max_{\mathbf{k}_{t+1}, \mathbf{n}_{t}, \mathbf{t}_{t}, \mathbf{g}_{t}} \{ \sum_{i} u_{i}\left(z_{i,t} k_{i,t}^\alpha n_{i,t}^{1-\alpha} + s_{i, t} - k_{i, t+1}, 1 - n_{i,t}\right) \\
    & + \beta E[A + \sum_{i} B_i \log k_{i, t+1} + \sum_{i} C_i \log z_{i, t+1} | I_t] \}
\end{align*}
\begin{equation}
    B_i = \frac{\alpha}{1-\alpha\beta}
\end{equation}
\begin{equation}
    C_i = \frac{1}{(1-\rho_i)}
\end{equation}
The optimal policies are:
\begin{equation}
    n_{i,t} = \frac{{\alpha - 1}}{{\alpha \beta \chi + \alpha - \chi - 1}}
\end{equation}
\begin{equation}
    c_{i,t} = (1 -\alpha \beta) (k_{i,t}^\alpha n_{i,t}^{1 - \alpha} z_{i,t} + s_{i,t})
\end{equation}
\begin{equation}
    k_{i,t+1} = \alpha \beta (k_{i,t}^{\alpha} n_{i,t}^{1 - \alpha} z_{i,t} + s_{i,t})
\end{equation}
Since \(c_{i,t}\) is the same for all \(i\), we get:
\begin{equation}
    s_{i, t} = \frac{1}{n} \sum_{j} z_{j, t} k_{j,t}^\alpha n_{j,t}^{1-\alpha} - z_{i,t} k_{i,t}^\alpha n_{i,t}^{1 - \alpha} 
\end{equation}
In other words, the government distributes the tax revenue in consideration of different production outputs.

% \subsection{Problem 2}

% The spline interpolation result is as follows:\\
% \begin{center}
% \includegraphics[scale=0.4]{spline_interpolation.png}
% \end{center}
% The matlab code is as follows:

% \begin{lstlisting}[language=Matlab]
% f = @(x) min(max(-1, 4*(x - 0.2)), 1);

% x = linspace(-1, 1, 100);
% y = arrayfun(f, x);
% xi = linspace(-1, 1, 1000);
% yi = interp1(x, y, xi, 'spline');

% figure;
% plot(x, y, 'o', xi, yi, '-');
% legend('Sample Points', 'Spline Interpolation');
% title('Spline Interpolation');
% xlabel('x');
% ylabel('f(x)');
% print('spline_interpolation', '-dpng');

% A = 1;
% beta = 0.99;
% delta = 0.025;
% alpha = 0.36;
% kmin = 0.06;
% kmax = 12;
% tol = 0.01;
% m = 300;

% k = kmin:kmin:kmax;

% [V, policy] = run_bellman(k, A, beta, delta, alpha, tol, m);
% [V_trivial, policy_trivial] = run_bellman(k, A, beta, 1, alpha, tol, m);

% figure;
% plot(k, V);
% title('Value Function');
% xlabel('Capital Stock k');
% ylabel('Value Function V(k)');
% print('value_function', '-dpng');

% figure;
% plot(k, policy, k, policy_trivial);
% title('Policy Function');
% xlabel('Capital Stock k');
% ylabel('Optimal k''');
% legend('delta = 0.025', 'delta = 1');
% print('policy_function', '-dpng');

% function [V, policy] = run_bellman(k, A, beta, delta, alpha, tol, m)
% n = length(k);
% V = zeros(1, n);

% for iter = 1:m
%     Vnew = zeros(1, n);
%     for i = 1:n
%         obj = @(j) (log(A*k(i)^alpha + (1-delta)*k(i) - k(j)) + beta*V(j));
        
%         jmax = find(k <= A*k(i)^alpha + (1-delta)*k(i), 1, 'last');
%         [val, ip] = max(arrayfun(obj, 1:jmax));
        
%         if A*k(i)^alpha + (1-delta)*k(i) - k(ip) <= 0
%             val = -Inf;
%         end
        
%         Vnew(i) = val;
%     end
    
%     if max(abs(V - Vnew)) < tol
%         break;
%     end
%     V = Vnew;
% end

% policy = zeros(1, length(k));
% for i = 1:n
%     obj = @(kp) -(log(A*k(i)^alpha + (1-delta)*k(i) - kp) + beta*interp1(k, V, kp, 'spline'));
%     [kp_opt, ~] = fminbnd(obj, 0, A*k(i)^alpha + (1-delta)*k(i));
%     policy(i) = kp_opt;
% end
% end
% \end{lstlisting}

% The results for the code is as follows:\\
% \begin{center}
% \includegraphics[scale=0.4]{value_function.png}\\
% \includegraphics[scale=0.4]{policy_function.png}
% \end{center}


\section{Part 2}

\subsection{}

\[
\max_{\{k_{t+1}, c_t, n_t, y_t, i_t\}} E^P \left( \sum_{t=0}^{\infty} \beta^{t} u(c_t, 1 - n_t) \,|\, I_0 \right) \text{subject to } y_t = i_t + c_t,\\
\]
\[
k_{t+1} = (1 - \delta) k_t + i_t,\\
\]
\[
y_t = z_t k_t^\alpha n_t^{1 - \alpha},\\
\]
\[
0 \leq n_t \leq 1,\\
\]
\[
k_0, z_0: \text{ given},\\
\]
\[
\{ z_t \}: \text{a random process},\\
\]

\[
P \equiv \text{\{subjective probability distribution at the individual level or} \\
\]
\[
\text{objective distribution in equilibrium (rational expectations)\}}.
\]

% \subsection{}
% For any function $g(\cdot, \cdot) \in S_{K \times Z}$, any $I_{t+1}$-measurable random variable $z_{t+1}$ distributed on $Z \equiv [z_{\text{min}}, z_{\text{max}}]$, and any realized value of $z_t$, there exists a continuous and bounded function $h(\cdot, \cdot) \in S_{K \times Z}$ such that for all $t \geq 0$, 
% \begin{equation}
%     E[g(k_{t+1},z_{t+1})|I_t] = h(k_{t+1}, z_t) \in S_{K \times Z}
% \end{equation}

% \subsection{}
% The transformation $T(g(k_{t+1},z_{t+1}))$ is given by:
% \begin{equation}
%     T(g(k_{t+1},z_{t+1})) = \max_{k_{t+1},n_t} u\left(z_t k_t^\alpha n_t^{1-\alpha} + (1-\delta) k_t - k_{t+1}, 1 - n_t\right) + \beta E[g(k_{t+1},z_{t+1})|I_t]
% \end{equation}
% By applying the Feller property, the transformation can be rewritten as:
% \begin{equation}
%     T(g(k_{t+1},z_{t+1})) = \max_{k_{t+1},n_t} u\left(z_t k_t^\alpha n_t^{1-\alpha} + (1-\delta) k_t - k_{t+1}, 1 - n_t\right) + \beta h(k_{t+1},z_{t})
% \end{equation}
% Let $(k^*_{t+1},n^*_t) \equiv F(k_t)$ be the optimal choice and set
% \begin{equation}
%     u\left(z_t k_t^\alpha (n^*_t)^{1-\alpha} + (1-\delta) k_t - k^*_{t+1}, 1 - n^*_t\right) + \beta h(k^*_{t+1},z_{t})
% \end{equation}
% as the maximum achievable value given the parameter $k_t$. Since $h(\cdot, \cdot) \in S_{K \times Z}$, $h$ is continuous and bounded. $u$ is trivially continuous and bounded.

% Then, by the maximum theorem, the expression
% \begin{equation}
%     u\left(z_t k_t^\alpha (n^*_t)^{1-\alpha} + (1-\delta) k_t - k^*_{t+1}, 1 - n^*_t\right) + \beta h(k^*_{t+1},z_{t})
% \end{equation}
% is continuous with respect to the parameter $k_t$. In other words, the Bellman operator
% \begin{equation}
%     T(g(k_{t+1},z_{t+1})) = \max_{k_{t+1},n_t} u\left(z_t k_t^\alpha n_t^{1-\alpha} + (1-\delta) k_t - k_{t+1}, 1 - n_t\right) + \beta E[g(k_{t+1},z_{t+1})|I_t]
% \end{equation}
% is a well-defined mapping from $S_{K \times Z}$ onto $S_{K \times Z}$.


% \subsection{}
% We can briefly expand Blackwell's Sufficiency Theorem for the two-dimensional case. For all functions $f, g \in S_{K \times Z}$, if $f$ and $g$ satisfy
% \begin{equation}
%     \forall t \in K \times Z, \ f(t) \leq g(t),
% \end{equation}
% then
% \begin{equation}
%     \forall t \in K \times Z, \ (Tf)(k) \leq (Tg)(g).
% \end{equation}

% Furthermore, for any function $f \in S_{K \times Z}$ and any non-negative scalar $a \geq 0$, we have

% \begin{align}
%     T((g+a)(k_{t+1},z_{t+1})) &= \max_{k_{t+1},n_t} u\left(z_t k_t^\alpha n_t^{1-\alpha} + (1-\delta) k_t - k_{t+1}, 1 - n_t\right) \\
%     &\quad + \beta E[(g+a)(k_{t+1},z_{t+1})|I_t] \\
%     &= \max_{k_{t+1},n_t} u\left(z_t k_t^\alpha n_t^{1-\alpha} + (1-\delta) k_t - k_{t+1}, 1 - n_t\right) \\
%     &\quad + \beta E[g(k_{t+1},z_{t+1})|I_t] + \beta \cdot a \\
%     &= T(g(k_{t+1},z_{t+1})) + \beta \cdot a,
% \end{align}

% where $0 < \beta < 1$. \\

% For all $t \in K \times Z$, we have
% \begin{equation}
%     |f(t) - g(t)| \leq ||f - g||
% \end{equation}
% Therefore,
% \begin{align*}
%     \forall t \in K \times Z, \ &f(t) \leq g(t) + ||f - g|| \\
%     \text{or} \ &g(t) \leq f(t) + ||f - g||
% \end{align*}
% This implies that
% \begin{align}
%     Tf &\leq T(g + ||f - g||) \leq Tg + \beta ||f - g|| \\
%     \text{or} \ &Tg \leq T(f + ||f - g||) \leq Tf + \beta ||f - g||
% \end{align}
% Thus, we conclude that
% \begin{equation}
%     ||Tf - Tg|| \leq \beta ||f - g||
% \end{equation}\\
% and $T$ is a contraction mapping with modulus $\beta$ of $S_{K \times Z}$ onto $S_{K \times Z}$.


% \subsection{}
% The Bellman equation for this problem is written as:
% \[ V(k_t, z_t) = \max_{k_{t+1}, n_t} \left\{ u\left(z_t k_t^\alpha n_t^{1-\alpha} + (1-\delta) k_t - k_{t+1}, 1 - n_t\right) + \beta E[V(k_{t+1}, z_{t+1}) | I_t] \right\} \]

% \[\text{subject to } k_t, z_t \text{ given}\]
% \[\text{and } I_t \text{ given} \]

% From problem 4, we know that the Bellman operator $T$ is a contraction mapping of $S_{K \times Z}$ onto $S_{K \times Z}$ with modulus $\beta$. 
% Therefore, the Bellman equation has a unique solution $V^*$ in $S_{K \times Z}$. 

\subsection{}

The first-order condition and differentiation under conditional expectation sign dictates:
\begin{equation}
    \partial{k_{t+1}}:  -u_c\left(c_t, 1 - n_t\right) + \beta E[V_k(k_{t+1}, z_{t+1}) | I_t] = 0
\end{equation}
\begin{equation}
    \partial{n_t}: -u_n\left(c_t, 1 - n_t\right) + \beta E[V_k(k_{t+1}, z_{t+1}) z_t (k_t)^{\alpha}(1-\alpha)n_t^{-\alpha}| I_t] = 0
\end{equation}
where $u_c$ is the marginal utility of consumption, and $V_k$ is the partial derivative of $V$ with respect to $k$. \\

The envelope condition dictates:
\begin{equation}
    \partial{k_t}: V_k(k_t, z_t) = u_c\left(c_t, 1 - n_t\right) \cdot \left(z_t \alpha k_t^{\alpha - 1} n_t^{1-\alpha} + 1 - \delta\right)
\end{equation}

Combining the above two equations, we have:
\begin{equation}
    -u_c\left(c_t, 1 - n_t\right) + \beta E[u_c\left(c_{t+1}, 1 - n_{t+1}\right) \cdot \left(z_{t+1} \alpha k_{t+1}^{\alpha - 1} n_{t+1}^{1-\alpha} + 1 - \delta\right) | I_t] = 0
\end{equation}
\begin{equation}
    \frac{1}{c_t} = \beta E\left[\frac{1}{c_{t+1}} \cdot \left(z_{t+1} \alpha k_{t+1}^{\alpha - 1} n_{t+1}^{1-\alpha} + 1 - \delta\right) | I_t\right]
\end{equation}
\begin{equation}
    \frac{1}{c_t} =  \frac{\beta}{c_{t+1}} \left( \alpha k_{t+1}^{\alpha - 1} n_{t+1}^{1-\alpha} E\left[z_{t+1} | I_t\right] + \left(1 - \delta\right) \right)
\end{equation}
From (19), we get:
\begin{equation}
    \frac{\chi}{1-n_t} = \beta E\left[\frac{1}{c_{t+1}} \cdot \left(z_{t+1} \alpha k_{t+1}^{\alpha - 1} n_{t+1}^{1-\alpha} + 1 - \delta\right)  \cdot z_t (k_t)^{\alpha}(1-\alpha)n_t^{-\alpha} | I_t\right]
\end{equation}
\begin{equation}
    \frac{\chi}{1-n_t} = \frac{1}{c_t} \cdot z_t (k_t)^{\alpha}(1-\alpha)n_t^{-\alpha}
\end{equation}
For capital accumulation, we have:
\begin{equation}
    k_{t+1} = z_t k_t^\alpha n_t^{1-\alpha} + (1-\delta) k_t - c_t
\end{equation}

\subsection{}

Suppose \(V(k_t, z_t) = A + B \log k_t + C \log z_t\), and for \(\delta = 1.0\), we have:
\[\frac{1}{c_t} =  \frac{\beta}{c_{t+1}} \alpha k_{t+1}^{\alpha - 1} n_{t+1}^{1-\alpha} E\left[z_{t+1} | I_t\right] \]
\[k_{t+1} = z_t k_t^\alpha n_t^{1-\alpha} - c_t\]
Bellman equation is written as:\\
\begin{align*}
    A +  B \log k_t + C \log z_t = & \max_{k_{t+1}, n_t} \{ \log\left(z_t k_t^\alpha n_t^{1-\alpha} - k_{t+1}\right)\left( 1 - n_t\right)^\chi \\
    & + \beta E[A + B \log k_{t+1} + C \log z_{t+1} | I_t] \}
\end{align*}

The first order condition with respect to \(k_{t+1}\) is:
\begin{equation}
   0 = \frac{{B \cdot \beta}}{{k_{t1}}} - \frac{1}{{k_t^{\alpha} \cdot n_t^{(1 - \alpha)} \cdot z_t - k_{t1}}}
\end{equation}

The first order condition with respect to \(n_t\) is:
\begin{equation}
    0 = -\frac{{\chi}}{{1 - n_t}} + \frac{{k_t^{\alpha} \cdot n_t^{(1 - \alpha)} \cdot z_t \cdot (1 - \alpha)}}{{n_t \cdot (k_t^{\alpha} \cdot n_t^{(1 - \alpha)} \cdot z_t - k_{t1})}}
\end{equation}

Suppose the optimal choice is \(k^*_{t+1}\) and \(n^*_t\), then we have:

\begin{equation}
    k^*_{t+1} = \frac{\beta B}{1 + \beta B} z_t k_t^\alpha n_t^{1-\alpha}
\end{equation}

\begin{equation}
    n^*_t = \frac{{\alpha - 1}}{{\alpha \beta \chi + \alpha - \chi - 1}}
\end{equation}

\begin{align*}
    A +  B \log k_t + C \log z_t = & \log\left(z_t k_t^\alpha n_t^{* 1-\alpha} - k^*_{t+1}\right) + \chi \log\left( 1 - n^*_t\right) \\
    & + \beta A + \beta B \log k^*_{t+1} + \beta C \rho \log z_t
\end{align*}

Using \(k^*_{t+1}\) we have:
\begin{align*}
    A +  B \log k_t + C \log z_t &= \beta ( A + B \log ( \frac{B \beta k_t^\alpha n_t^{* 1 - \alpha} z_t}{B \beta + 1} ) \\
    &+ C \rho \log(z_t) ) + \chi \log\left( 1 - n^*_t\right) + \log ( k_t^\alpha n_t^{1 - \alpha} z_t - \frac{B \beta k_t^\alpha n_t^{1 - \alpha} z_t}{B \beta + 1} )
\end{align*}

Then, the coefficients are:\\
\begin{equation}
    A = \frac{(1-\alpha)\log(n^*_t) + (1 - \alpha\beta)\chi\log(1 - n^*_t) + \alpha\beta\log(\alpha\beta) +(1- \alpha\beta)\log(1-\alpha\beta)}{(1 - \alpha\beta)(1 - \beta)}
\end{equation}

\begin{equation}
    B = \frac{\alpha}{1-\alpha\beta}
\end{equation}

\begin{equation}
    C = \frac{1}{(1-\alpha\beta)(1-\rho)}
\end{equation}

The optimal policy functions are:\\
\begin{equation}
    n_t = \frac{{\alpha - 1}}{{\alpha \beta \chi + \alpha - \chi - 1}}
\end{equation}

\begin{equation}
    c_t = (1 -\alpha \beta) k_t^\alpha n_t^{1 - \alpha} z_t
\end{equation}

\begin{equation}
    k_{t+1} = \alpha \beta k_{t}^{\alpha} n_{t}^{1 - \alpha} z_{t}
\end{equation}

\subsection{}
Bellman equation is written as:\\
\begin{equation}
    V(k_t, z_t) = \max_{k_{t+1}, n_t} \left\{ u\left(z_t k_t^\alpha n_t^{1-\alpha}-i_t, 1 - n_t\right) + \beta E[V(k_{t+1}, z_{t+1}) | I_t] \right\}
\end{equation}
For simplification, we maximize w.r.t. \(i_t\), since determining \(i_t\) is equivalent to determining \(k_{t+1}\). \\ \\
Suppose \(V(k_t, z_t) = A + B \log k_t + C \log z_t\), then we have:
\begin{align*}
    A +  B \log k_t + C \log z_t = & \max_{i_{t}, n_t} \{ \log\left(z_t k_t^\alpha n_t^{1-\alpha}-i_t\right) + \chi\log\left( 1 - n_t\right) \\
    &  + \beta A + \beta B \log (k_{t})^{1-\delta}(i_t)^\delta + \beta \rho C \log z_{t} \}
\end{align*}
First order condition with respect to \(i_t\):\\
\begin{equation}
    0 = \frac{{B \cdot \beta \cdot \delta}}{i_t} - \frac{1}{{-i_t + k_t^{\alpha} \cdot n_t^{(1 - \alpha)} \cdot z_t}}
\end{equation}\\
First order condition with respect to \(n_t\):\\
\begin{equation}
    0 = -\frac{{\chi}}{{1 - n_t}} + \frac{{k_t^{\alpha} \cdot n_t^{(1 - \alpha)} \cdot z_t \cdot (1 - \alpha)}}{{n_t \cdot (-i_t + k_t^{\alpha} \cdot n_t^{(1 - \alpha)} \cdot z_t)}}
\end{equation}\\
After solving the above two equations, we have:\\
\begin{equation}
    B = \frac{\alpha}{1-\beta(1-\delta)-\alpha\beta\delta}
\end{equation}
\begin{equation}
    C = \frac{1}{(1-\rho\beta)}
\end{equation}

\begin{equation}
    i_t^* = \frac{B\beta\delta k_t^\alpha n_t^{1 - \alpha}z_t}{B\beta\delta + 1}
\end{equation}

\begin{equation}
    n_t^* = \frac{-B\alpha\beta\delta + B\beta\delta - \alpha + 1}{-B\alpha\beta\delta + B\beta\delta - \alpha + \chi + 1}
\end{equation}\\

Therefore, for control variables, we have:\\
\begin{equation}
    c_t = \frac{k_t^\alpha n_t^{1-\alpha} z_t (-\alpha\beta\delta + \beta\delta - \beta + 1)}{\beta\delta - \beta + 1}
\end{equation}

\begin{equation}
    n_t = \frac{\alpha\beta\delta - \alpha\beta + \alpha - \beta\delta + \beta - 1}{\alpha\beta\chi\delta + \alpha\beta\delta - \alpha\beta + \alpha - \beta\chi\delta + \beta\chi - \beta\delta + \beta - \chi - 1}
\end{equation}\\

For state variables, we have:\\
\begin{equation}
    k_{t+1} = k_t^{1-\delta} \left(\frac{\alpha\beta\delta k_t^\alpha n_t^{1-\alpha} z_t}{\beta\delta - \beta + 1}\right)^\delta
\end{equation}


\hspace{1em} The law of motion for capital can be written as \(k_{t+1} = k_t(\frac{i_t}{k_t})^{1-\delta}\) and a few observations can be made.
First, the investment should be always nonzero in order to maintain the capital stock above zero in the next period.
This is similar to the Brock-Mirman model with \(\delta = 1\), where investment should be positive for capital accumulation.
Second, unlike the Brock-Mirman model, the investment is not additive to the capital stock, but rather multiplied and scaled.
The capital in the next time step is invariant to scale in capital and investment.
The equation also suggests that the investment should be larger than the current capital stock \(k_t\), otherwise the capital stock begins to shrink.
The Brock-Mirman model differs in that the investment should larger than \(\delta k_t\) for the capital stock to grow.
Third, the concavity of this investment-capital ratio implies diminishing marginal return of investment,
whereas the Brock-Mirman model is constant.
This relation is referred to as \textit{the capital adjustment cost} in our survey of the literature.\\

The business cycle implications are as follows:
\begin{equation}
    \mathrm{Var}[\log i_t^*] = \mathrm{Var}[\log c_t^*] = \mathrm{Var}[\log y_t^*]
\end{equation}
This result is the same as the Brock-Mirman model with \(\delta = 1\).

\subsection{}
The Bellman equation for this problem is written as:
\begin{equation}
    V(k_t, z_t) = \max_{k_{t+1}, n_t} \left\{ \log\left(z_t k_t^\alpha n_t^{1-\alpha}-i_t-\chi (n_t)^{1+\omega}\right) + \beta E[V(k_{t+1}, z_{t+1}) | I_t] \right\}
\end{equation}
For simplification, we maximize w.r.t. \(i_t\), since determining \(i_t\) is equivalent to determining \(k_{t+1}\). \\ \\
Suppose \(V(k_t, z_t) = A + B \log k_t + C \log z_t\), then we have:
\begin{align*}
    A +  B \log k_t + C \log z_t = & \max_{i_{t}, n_t} \{ \log\left(z_t k_t^\alpha n_t^{1-\alpha}-i_t-\chi (n_t)^{1+\omega}\right) \\
    &  + \beta A + \beta B (1-\delta) \log (k_{t}) + \beta B \delta \log(i_t) + \beta \rho C \log z_{t} \}
\end{align*}
First order condition with respect to \(i_t\):\\
\begin{equation}
    i_t^* = \frac{{B \cdot \beta \cdot \delta \cdot (-\chi \cdot n_t^{(\omega + 1)} + k_t^{\alpha} \cdot n_t^{(1 - \alpha)} \cdot z_t)}}{{B \cdot \beta \cdot \delta + 1}}
\end{equation}
First order condition with respect to \(n_t\):\\
\begin{equation}
    n_t^* = \left(\frac{1-\alpha}{\chi (1+\omega)}\right)^{\frac{1}{\omega + \alpha}} \left(k_t^\alpha z_t\right)^{\frac{1}{\omega + \alpha}}
\end{equation}
\begin{equation}
    y_t^* = \left(\frac{1-\alpha}{\chi (1+\omega)}\right)^{\frac{1-\alpha}{\alpha + \omega}} \left(k_t^\alpha z_t\right)^{\frac{\omega + 1}{\omega + \alpha}}
\end{equation}

Suppose \(\overline{a} = \left(\frac{1-\alpha}{\chi (1+\omega)}\right)^{\frac{1}{\alpha + \omega}}\), then we have:
\begin{equation}
\log(y_t^* -\chi \cdot n_t^{(\omega + 1)}) = \log\left(\overline{a}^{1-\alpha} - \chi \overline{a}^{1+\omega}\right)+ \frac{\alpha(\omega + 1)}{\omega + \alpha} \log (k_{t}) + \frac{\omega + 1}{\omega + \alpha} \log(z_t)
\end{equation}


\begin{align*}
    A +  B \log k_t + C \log z_t = & -\log\left(1+B\beta\delta\right) + \log\left(\overline{a}^{1-\alpha} - \chi \overline{a}^{1+\omega}\right) \\
    & + \frac{\alpha(\omega + 1)}{\omega + \alpha} \log (k_{t}) + \frac{\omega + 1}{\omega + \alpha} \log(z_t) \\
    &  + \beta A + \beta B (1-\delta) \log (k_{t}) + \beta B \delta \log(i_t) + \beta \rho C \log z_{t}
\end{align*}


\begin{align*}
    A +  B \log k_t + C \log z_t = &\left(B\beta\delta\right) \log\left(B\beta\delta\right)-\left(1+B\beta\delta\right) \log\left(1+B\beta\delta\right) + \beta A\\
    &\left(1+B\beta\delta\right)\log\left(\overline{a}^{1-\alpha} - \chi \overline{a}^{1+\omega}\right)+ \left(1+B\beta\delta\right)\frac{\alpha(\omega + 1)}{\omega + \alpha} \log (k_{t}) + \left(1+B\beta\delta\right)\frac{\omega + 1}{\omega + \alpha} \log(z_t) \\
    &\beta B (1-\delta) \log (k_{t}) + \beta \rho C \log z_{t}
\end{align*}

\begin{equation} 
    B = \frac{\alpha(-\omega - 1)}{\alpha\beta\delta\omega + \alpha\beta - \alpha - \beta\delta\omega + \beta\omega - \omega} 
\end{equation}

\begin{equation}
   C = \frac{{(\omega + 1)  (\alpha  \beta  \delta - \alpha  \beta + \alpha + \beta  \delta  \omega - \beta  \omega + \omega)}}{{(\alpha  \beta  \rho - \alpha + \beta  \omega  \rho - \omega)  (\alpha  \beta  \delta  \omega + \alpha  \beta - \alpha - \beta  \delta  \omega + \beta  \omega - \omega)}}
\end{equation}

\begin{equation}
    i_t^* = \frac{{\alpha  \beta  \delta  (\omega + 1)  (k_t^{\alpha}  n_t^{(1 - \alpha)}  z_t - \chi  n_t^{(\omega + 1)})}}{{\alpha  \beta  \delta  - \alpha  \beta + \alpha + \beta  \delta  \omega - \beta  \omega + \omega}}
\end{equation}

The optimal household polices are:
\begin{equation}
    n_t = \left(\frac{1-\alpha}{\chi (1+\omega)}\right)^{\frac{1}{\omega + \alpha}} \left(k_t^\alpha z_t\right)^{\frac{1}{\omega + \alpha}}
\end{equation}

\begin{equation}
    k_{t+1} = k_t^{\delta} \left(\frac{{\alpha  \beta  \delta  (\omega + 1)  (k_t^{\alpha}  n_t^{(1 - \alpha)}  z_t - \chi  n_t^{(\omega + 1)})}}{{\alpha  \beta  \delta  - \alpha  \beta + \alpha + \beta  \delta  \omega - \beta  \omega + \omega}}\right)^{1-\delta}
\end{equation}

\begin{equation}
    c_t = y_t - i_t
\end{equation}

\section{Part 3}

\subsection{Problem 1}
The representative consumer’s optimum problem is:
\[
    \max_{\{A_{t+1}\}_{t=0}^{\infty}} E\left[\sum_{t=0}^{\infty} \beta^t u(A_t + w_t - \frac{A_{t+1}}{1+r}) | I_0 \right]
\]
subject to:
\[
    A_0, w_0 \text{ given}    
\]
\[
    0 \leq A_{t+1} \leq  (1+r)(A_t + w_t) \quad \forall t \geq 0
\]
\[
    \{w_t\}_{t=0}^{\infty} \text{ is a random process}    
\]
The Bellman equation is:
\[
    V(A_t, w_t) = \max_{A_{t+1}} \left\{ u(A_t + w_t - \frac{A_{t+1}}{1+r}) + \beta E[V(A_{t+1}, w_{t+1}) | I_t] \right\}
\]
\subsection{Problem 2}
The first order condition with differentiation under conditional expectation sign dictates:
\begin{equation}
    \partial_{A_{t+1}} : \frac{-1}{1+r}u'(A_t + w_t - \frac{A_{t+1}}{1+r}) + \beta E\left[\frac{\partial V(A_{t+1}, w_{t+1})}{\partial A_{t+1}} | I_t\right] = 0
\end{equation}
The envelope condition is:
\begin{equation}
    \partial_{A_t} V(A_t, w_t) = u'(A_t + w_t - \frac{A_{t+1}}{1+r})
\end{equation}
The Euler equation is:
\begin{equation}
    \frac{1}{1+r}u'(A_t + w_t - \frac{A_{t+1}}{1+r}) = \beta E\left[u'(A_{t+1} + w_{t+1} - \frac{A_{t+2}}{1+r}) | I_t\right]
\end{equation}
This equation becomes:
\begin{equation}
    E\left[\frac{u'(c_{t+1})}{u'(c_t)} | I_t\right] = \frac{1}{\beta (1+r)}
\end{equation}

\subsection{Problem 3}
Suppose \(\beta = \frac{1}{1+r}\). Then the Euler equation becomes:
\begin{equation}
    E\left[u'(c_{t+1}) | I_t\right] = u'(c_t)
\end{equation}
Taylor expansion of \(u'(c_{t+1})\) around \(c_t\) gives:
\begin{equation}
    E\left[u'(c_{t+1}) | I_t\right] = u'(c_t) + u''(c_t)(E\left[c_{t+1} | I_t\right] - c_t) + E\left[O( (c_{t+1}-c_t)^2 )| I_t\right]
\end{equation}
The Euler equation becomes:
\begin{equation}
    E\left[c_{t+1} | I_t\right]  = c_t - \frac{1}{u''(c_t)}E\left[O( (c_{t+1}-c_t)^2 )| I_t\right]
\end{equation}
Suppose \(c_t \approx c_{t+1}\), then \(O( (c_{t+1}-c_t)^2 ) \rightarrow 0\). Therefore, the Euler equation becomes:
\begin{equation}
    E\left[c_{t+1} | I_t\right]  = c_t
\end{equation}
This implies that the optimal consumption is martingale and the following holds:
\begin{equation}
    E\left[c_{t+k} | I_t\right]  = c_t
\end{equation}
Since
\begin{equation}
    E\left[c_{t+k} | I_t\right] = E\left[E\left[c_{t+k} | I_{t+k-1}\right] | I_t\right] = E\left[c_{t+k-1} | I_t\right] = \cdots = E\left[c_{t+1} | I_t\right] = c_t
\end{equation}
The intertemporal budget constraint with no-Ponzi scheme is:
\begin{equation}
    \sum_{t=0}^{\infty} \frac{1}{(1+r)^t} c_t = A_0 + \sum_{t=0}^{\infty} \frac{1}{(1+r)^t} w_t
\end{equation}
Take expectation on both sides, we have:
\begin{equation}
    \sum_{t=0}^{\infty} \frac{1}{(1+r)^t} E\left[c_t | I_0\right] = A_0 + \sum_{t=0}^{\infty} \frac{1}{(1+r)^t} E\left[w_t | I_0\right]
\end{equation}
Since \(E[c_t|I_0] = E[c_0|I_0] = c_0\), we have:
\begin{equation}
    c_0 = \frac{r}{1+r}\left( A_0 + \sum_{t=0}^{\infty} \frac{1}{(1+r)^t} E\left[w_t|I_0\right] \right)
\end{equation}

This implies that if the discount rate is equal to the interest rate, the consumption is equal to the expected present discounted sum of lifetime income.
The expected return from the initial asset price also affects the consumption

\subsection{Problem 4}
\begin{equation}
    u'(c) = -(\overline{c} - c)
\end{equation}
Therefore, the Euler equation becomes:
\begin{equation}
    E\left[c_{t+1} | I_t\right] = \frac{c_t}{\beta (1+r)} + \frac{\overline{c}(\beta (1+r) - 1)}{\beta (1+r)}
\end{equation}
If we assume that \(\beta = \frac{1}{1+r}\), then the optimal consumption is martingale:
\begin{equation}
    E\left[c_{t+1} | I_t\right] = c_t
\end{equation}
This can be extended to time shift of \(k\):
\begin{equation}
    E\left[c_{t+k} | I_t\right] = c_t
\end{equation}
Therefore, the expected life time consupmtion is:
\begin{equation}
    \sum_{t=0}^{\infty} \frac{1}{(1+r)^t}E\left[c_t|I_0\right] = \frac{1+r}{r} c_0
\end{equation}
From the intertemporal budget constraint and no-Ponzi scheme, we have:
\begin{equation}
    \sum_{t=0}^{\infty} \frac{1}{(1+r)^t} E\left[c_t | I_0\right] = A_0 + \sum_{t=0}^{\infty} \frac{1}{(1+r)^t} E\left[w_t | I_0\right]
\end{equation}
Therefore, the consumption is:
\begin{equation}
    c_0 = \frac{r}{1+r}\left( A_0 + \sum_{t=0}^{\infty} \frac{1}{(1+r)^t} E\left[w_t|I_0\right] \right)
\end{equation}
In other words, if the discount rate is equal to the interest rate, the consumption is determined by the expected present value of permanent income and the inital asset.

\subsection{Problem 5}
The Bellman equation is:
\[
    V(A_t) = \max_{A_{t+1}} \left\{ u(A_t - \frac{A_{t+1}}{R_{t+1}}) + \beta E[V(A_{t+1}) | I_t] \right\}
\]
If \( u(c_t) = \frac{c_t^{1-\gamma}}{1-\gamma} \), then the Euler equation becomes:
\begin{equation}
    \frac{1}{R_{t+1}} c_t^{-\gamma} = \beta E\left[c_{t+1}^{-\gamma} | I_t\right]
\end{equation}
Suppose \(V(A_t) = B \frac{A_t^{1-\gamma}}{1-\gamma} \), then the Bellman equation becomes:
\begin{equation}
    B \frac{A_t^{1-\gamma}}{1-\gamma} = \max_{A_{t+1}} \left\{ \frac{1}{1-\gamma}(A_t - \frac{A_{t+1}}{R_{t+1}})^{1-\gamma} + \beta E\left[B \frac{A_{t+1}^{1-\gamma}}{1-\gamma} | I_t\right] \right\}
\end{equation}
This is equivalent to:
\begin{equation}
    B \frac{A_t^{1-\gamma}}{1-\gamma} = \max_{c_t} \left\{ \frac{1}{1-\gamma}c_t^{1-\gamma} + \beta E\left[B \frac{(R_{t+1}(A_t - c_t))^{1-\gamma}}{1-\gamma} | I_t\right] \right\}
\end{equation}
\begin{equation}
    B \frac{A_t^{1-\gamma}}{1-\gamma} = \max_{c_t} \left\{ \frac{1}{1-\gamma}c_t^{1-\gamma} + \beta B \frac{(A_t - c_t)^{1-\gamma}}{1-\gamma} E\left[R_{t+1}^{1-\gamma} | I_t\right] \right\}
\end{equation}
Since \(R_{t+1}\) is unknown at time t and independent of \(I_t\), and the first order condition on \(c_t\) and we have:
\begin{equation}
    c_t^{-\gamma} = \beta B \cdot (A_t - c_t)^{-\gamma}E\left[R_{t+1}^{1-\gamma}\right]
\end{equation}
\begin{equation}
    c_t = \frac{A_t}{1 + \left(\beta B \cdot E\left[R_{t+1}^{1-\gamma}\right] \right)^{\frac{1}{\gamma}}}
\end{equation}
\begin{equation}
    B \frac{A_t^{1-\gamma}}{1-\gamma} = \frac{1}{1-\gamma} A_t^{1-\gamma}\left(1+(B \beta \cdot E\left[R_{t+1}^{1-\gamma}\right])^{\frac{1}{\gamma}}\right)^\gamma
\end{equation}
\begin{equation}
    B = \frac{1}{\left(1-\left(E\left[R_{t+1}^{1-\gamma}\right]\right)^{\frac{1}{\gamma}}\beta^{\frac{1}{\gamma}} \right)^\gamma}
\end{equation}
Therefore, the value function becomes:
\begin{equation}
    V(A_t) = \frac{1}{\left(1-\left(E\left[R_{t+1}^{1-\gamma}\right]\right)^{\frac{1}{\gamma}}\beta^{\frac{1}{\gamma}} \right)^\gamma} \frac{A_t^{1-\gamma}}{1-\gamma}
\end{equation}
The optimal consumption is:
\begin{equation}
    c_t = \left(1-\left(E\left[R_{t+1}^{1-\gamma}\right]\right)^{\frac{1}{\gamma}}\beta^{\frac{1}{\gamma}} \right) \cdot A_t
\end{equation}

\subsection{Problem 6}
The Bellman equation is:
\begin{equation}
    V(A_t) = \max_{c_{t}} \left\{ \frac{1}{1-\gamma}c_t^{1-\gamma} + \beta E[V(A_{t+1}) | I_t] \right\}
\end{equation}
The first order condition on \(A_{t+1}\) is:
\begin{equation}
    c_t^{-\gamma} = \beta E\left[ R_{t+1}  V'(A_{t+1}) | I_t\right]
\end{equation}
The Benveniste-Scheinkman theorem with uncertainty dictates:
\begin{equation}
    \partial_{A_t} V(A_t) = V'(A_t) = c_t^{-\gamma}
\end{equation}
Time shift the above equation by 1 period, and with the first order condition, we have:
\begin{equation}
    c_t^{-\gamma} = \beta E\left[ R_{t+1} c_{t+1}^{-\gamma} | I_t\right]
\end{equation}
The guess-and-verify method of linear policy \(c_{t+1}^{-\gamma} = a A_t + b w_t + c\) failed, since it is impossible to find appropriate \(b\).
The is due to the inability to link \(w_t\) and \(w_{t+1}\) together. \\
Try guess-and-verify of \(V(A_t) = B \frac{A_t^{1-\gamma}}{1-\gamma} \), then the Euler equation becomes:
\begin{equation}
    c_t^{-\gamma} = \beta B \cdot E\left[ R_{t+1}^{1-\gamma} (A_t + w_t - c_t)^{-\gamma} | I_t\right]
\end{equation}
Since \(R_{t+1}\) is independent of \(I_t\) and \(w_t\), we have:
\begin{equation}
    c_t^{-\gamma} = \beta B \cdot E\left[ R_{t+1}^{1-\gamma}\right] E\left[(A_t + w_t - c_t)^{-\gamma} | I_t\right]
\end{equation}

Guess-and-verify of \(V(A_t) = B \frac{A_t^{1-\gamma}}{1-\gamma} \) fails, since the right hand side is a multinomial of \(c_t\), and not a single term of the power of \(c_t\).
The value function, representing the maximum utility attainable in any given state over an infinite horizon, should be invariant to the i.i.d and \(I_t\) independent random variables in its closed form. 
In other words, the value function is constant with respect to \(w_t\) and \(R_{t+1}\).
This suggests the only possible functional form is \(V(A_t) = B \frac{A_t^{1-\gamma}}{1-\gamma} \).
Failure to find appropriate \(B\) implies that the value function cannot be expressed in a closed form. \\

\subsection{Problem 7}
The Euler equation is:
\begin{equation}
    c_t^{-\gamma} = \beta E\left[ R_{t+1}\right] E\left[c_{t+1}^{-\gamma}|I_t\right], \quad 
\end{equation}
In other words:
\begin{equation}
    E\left[\frac{c_{t+1}^{-\gamma}}{c_t^{-\gamma}}|I_t\right] = \frac{1}{\beta E\left[ R_{t+1}\right]}
\end{equation}
Taking log and applying Jensen's inequality, we have:
\begin{equation}
   \log \left( \frac{1}{\beta E\left[ R_{t+1}\right]} \right) = \log \left( E\left[\frac{c_{t+1}^{-\gamma}}{c_t^{-\gamma}}|I_t\right] \right) \geq -\gamma E\left[ \log(c_{t+1}) - \log(c_t) \right]
\end{equation}
In other words:
\begin{equation}
    \frac{1}{\gamma} \log \left( \beta E\left[ R_{t+1}\right] \right) \leq E\left[ \log(c_{t+1}) - \log(c_t) |I_t\right]
\end{equation}
To put into words, the long run growth of consumption is lower bounded by the log of the expected return. \\

Since the consumption is not martingale, this cannot be fit into the intertemporal budget constraint with no-Ponzi scheme as before.
However, for the further analysis of short term change of consumption, we conduct taylor expnasion on \(u'(c_{t+1})\) around \(c_t\):
\begin{equation}
    u'(c_t) = \beta R E_t \left[ u'(c_{t}) + u''(c_t)(c_{t+1} - c_t) + \frac{1}{2} u'''(c_t)(c_{t+1} - c_t)^2 + ... \right]
\end{equation}
Since \(\frac{u''(c_t)}{u'(c_t)} = -\gamma\) and \(\frac{u'''(c_t)}{u'(c_t)} = \gamma(\gamma+1)\), we have:
\begin{equation}
    E_t \left[ -\gamma(c_{t+1} - c_t) + \frac{1}{2} \gamma(\gamma+1)(c_{t+1} - c_t)^2 + ... \right] = \frac{1}{\beta E[R]} - 1
\end{equation}
Where if we assume that \(c_t \approx c_{t+1}\), then the above equation becomes:
\begin{equation}
    E_t \left[ c_{t+1} \right] = \frac{\beta E[R] - 1}{\gamma\beta E[R]} + c_t
\end{equation}
Therefore, if \(\beta E[R] > 1\), then consumption follows a submartingale process. 
Conversely, if \(\beta E[R] \leq 1\), consumption behaves as a supermartingale.
Also, \(\gamma\) affects its degree.
Depending on the value of \(\frac{\beta E[R] - 1}{\gamma\beta E[R]}\), the consumption will either increase or decrease over time.
However, when \(\beta E[R] = 1\), the consumption is martingale, and the permanent income hypothesis holds.

\end{document}

% Suppose \(V(A_t) = B \frac{A_t^{1-\gamma}}{1-\gamma} \), then the Bellman equation becomes:
% \begin{equation}
%     B \frac{A_t^{1-\gamma}}{1-\gamma} = \max_{c_t} \left\{ \frac{1}{1-\gamma}c_t^{1-\gamma} + \beta E\left[B \frac{(R_{t+1}(A_t + w_t - c_t))^{1-\gamma}}{1-\gamma} | I_t\right] \right\}
% \end{equation}

% The right hand side becomes:
% \begin{equation}
%     \frac{1}{1-\gamma}c_t^{1-\gamma} + \beta B \frac{(A_t + w_t - c_t)^{1-\gamma}}{1-\gamma} E\left[R_{t+1}^{1-\gamma} | I_t\right]
% \end{equation}
% This is because:
% \begin{equation}
%     E\left[(R_{t+1}(A_t + w_t - c_t))^{1-\gamma} | I_t\right] = E\left[R_{t+1}^{1-\gamma} \sum_{k=0}^{1-\gamma} \binom{1-\gamma}{k} (A_t - c_t)^{1-\gamma-k} (w_t)^{k} | I_t\right]
% \end{equation}


% By linearity of conditional expectation, we have:
% \begin{equation}
%     E\left[R_{t+1}^{1-\gamma} \sum_{k=0}^{1-\gamma} \binom{1-\gamma}{k} (A_t - c_t)^{1-\gamma-k} (w_t)^{k} | I_t\right] = \sum_{k=0}^{1-\gamma} \binom{1-\gamma}{k} (A_t - c_t)^{1-\gamma-k} E\left[w_t^{k}R_{t+1}^{1-\gamma}  | I_t\right]
% \end{equation}
% Since \(w_t\) is independent of \(R_{t+1}\), and \(w_t\) is independent of \(I_t\), we have:
% \begin{equation}
%     E\left[w_t^{k}R_{t+1}^{1-\gamma}  | I_t\right] = E\left[w_t^{k} | I_t\right] E\left[R_{t+1}^{1-\gamma}  | I_t\right] = E\left[w_t^{k}\right] E\left[R_{t+1}^{1-\gamma}  | I_t\right] = Ew \cdot Er
% \end{equation}
% Where we denote \(Ew = E\left[w_t^{k}\right]\) and \(Er = E\left[R_{t+1}^{1-\gamma}  | I_t\right]\). Therefore, the expectation becomes:
% \begin{equation}
%     E\left[(R_{t+1}(A_t + w_t - c_t))^{1-\gamma} | I_t\right] = \sum_{k=0}^{1-\gamma} \binom{1-\gamma}{k} (A_t - c_t)^{1-\gamma-k} Ew \cdot Er
% \end{equation}
